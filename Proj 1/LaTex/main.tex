


\documentclass[a4paper,titlepage,11pt,twosides,floatssmall]{mwrep}

%matlab style
    \usepackage{listings}
    \lstset{ 
    	language=Matlab,                		% choose the language of the code
    %	basicstyle=10pt,       				% the size of the fonts that are used for the code
    	numbers=left,                  			% where to put the line-numbers
    	numberstyle=\footnotesize,      		% the size of the fonts that are used for the line-numbers
    	stepnumber=1,                   			% the step between two line-numbers. If it's 1 each line will be numbered
    	numbersep=5pt,                  		% how far the line-numbers are from the code
    %	backgroundcolor=\color{white},  	% choose the background color. You must add \usepackage{color}
    	showspaces=false,               		% show spaces adding particular underscores
    	showstringspaces=false,         		% underline spaces within strings
    	showtabs=false,                 			% show tabs within strings adding particular underscores
    %	frame=single,	                			% adds a frame around the code
    %	tabsize=2,                				% sets default tabsize to 2 spaces
    %	captionpos=b,                   			% sets the caption-position to bottom
    	breaklines=true,                			% sets automatic line breaking
    	breakatwhitespace=false,        		% sets if automatic breaks should only happen at whitespace
    	escapeinside={\%*}{*)}  ,        		% if you want to add a comment within your code
    	extendedchars=true,
    literate={á}{{\'a}}1 {ã}{{\~a}}1 {é}{{\'e}}1
    }



\usepackage[left=2.5cm,right=2.5cm,top=2.5cm,bottom=2.5cm]{geometry}
\usepackage[OT1]{fontenc}
\usepackage{polski}
\usepackage{amsmath}
\usepackage{amsfonts}
\usepackage{amssymb}
\usepackage{graphicx}
\usepackage{url}
\usepackage{tikz}
\usetikzlibrary{arrows,calc,decorations.markings,math,arrows.meta}
\usepackage{rotating}
\usepackage[percent]{overpic}
% \usepackage[cp1250]{inputenc}
\usepackage[utf8]{inputenc}
\usepackage{float}
\usepackage{xcolor}
\usepackage{pgfplots}
\usetikzlibrary{pgfplots.groupplots}
\usepackage{listings}
\usepackage{matlab-prettifier}
\usepackage{enumitem,amssymb}


\usepackage{hyperref}


\definecolor{szary}{rgb}{0.95,0.95,0.95}
\usepackage{siunitx}
\usepackage{placeins}
\sisetup{detect-weight,exponent-product=\cdot,output-decimal-marker={,},per-mode=symbol,binary-units=true,range-phrase={-},range-units=single}
\SendSettingsToPgf
%konfiguracje pakietu listings
\lstset{
	backgroundcolor=\color{szary},
	frame=single,
	breaklines=true,
}
\lstdefinestyle{customlatex}{
	basicstyle=\footnotesize\ttfamily,
	%basicstyle=\small\ttfamily,
}
\lstdefinestyle{customc}{
	breaklines=true,
	frame=tb,
	language=C,
	xleftmargin=0pt,
	showstringspaces=false,
	basicstyle=\small\ttfamily,
	keywordstyle=\bfseries\color{green!40!black},
	commentstyle=\itshape\color{purple!40!black},
	identifierstyle=\color{blue},
	stringstyle=\color{orange},
}
\lstdefinestyle{custommatlab}{
	captionpos=t,
	breaklines=true,
	frame=tb,
	xleftmargin=0pt,
	language=matlab,
	showstringspaces=false,
	%basicstyle=\footnotesize\ttfamily,
	basicstyle=\scriptsize\ttfamily,
	keywordstyle=\bfseries\color{green!40!black},
	commentstyle=\itshape\color{purple!40!black},
	identifierstyle=\color{blue},
	stringstyle=\color{orange},
}

%wymiar tekstu (bez �ywej paginy)
\textwidth 160mm \textheight 247mm

%ustawienia pakietu pgfplots
\pgfplotsset{
tick label style={font=\scriptsize},
label style={font=\small},
legend style={font=\small},
title style={font=\small}
}

\def\figurename{Rys.}
\def\tablename{Tab.}

%konfiguracja liczby p�ywaj�cych element�w
\setcounter{topnumber}{0}%2
\setcounter{bottomnumber}{3}%1
\setcounter{totalnumber}{5}%3
\renewcommand{\textfraction}{0.01}%0.2
\renewcommand{\topfraction}{0.95}%0.7
\renewcommand{\bottomfraction}{0.95}%0.3
\renewcommand{\floatpagefraction}{0.35}%0.5

\begin{document}
\frenchspacing
\pagestyle{plain} %zmieniłem z uheadings żeby były numery stron

%strona tytu�owa
\title{\bf\vskip 0.1cm}
\author{Hubert Kowalski, Rafał Wiercioch, Kuba Świerlikowski}
\date{2020}

\makeatletter
\renewcommand{\maketitle}{\begin{titlepage}
\begin{center}{\LARGE {\bf
Wydział Elektroniki i Technik Informacyjnych}}\\
\vspace{0.4cm}
{\LARGE {\bf Politechnika Warszawska}}\\
\vspace{0.3cm}
\end{center}
\vspace{5cm}
\begin{center}
{\bf \LARGE Projektowanie układów sterowania \vskip 0.1cm}
\end{center}
\vspace{1cm}
\begin{center}
{\bf \LARGE Projekt I \vskip 0.1cm}
\end{center}
\vspace{1cm}
\begin{center}
{\bf \LARGE \@title}
\end{center}
% https://www.overleaf.com/project/5e7608384f07e60001a452c0
\vspace{2cm}
\begin{center}
{\bf \Large \@author \par}
\end{center}
\vspace*{\stretch{6}}
\begin{center}
\bf{\large{Warszawa, \@date\vskip 0.1cm}}
\end{center}
\end{titlepage}
}


\maketitle
\tableofcontents
\chapter{Sprawdzenie poprawności wartości $U_{\mathrm{PP}}$, $Y_{\mathrm{PP}}$}
    \section{Implementacja}
        Implementacja zadania w środowisku Matlab jest w pliku \texttt{zad1\_sprawdzenie\_punktu\_pracy}.
    \section{Sprawdzenie poprawności wartości $U_{\mathrm{PP}}$,$Y_{\mathrm{PP}}$}
        Symulacja obiektu z wartościami wejściowymi równymi U=$U_{\mathrm{PP}}$ i Y=$Y_{\mathrm{PP}}$  dała odpowiedź w postaci sygnału równego Y=$Y_{\mathrm{PP}}$ co wskazuje na poprawność wartości punktu pracy. Symulacja widoczna na rysunkach poniżej:

 % \iffalse
\begin{tikzpicture}
\begin{axis}[
     width=14cm,
    height=10cm,
    title={Rys 1.1},
    xlabel={Próbki},
    ylabel={Wartość},
    xmin=0, xmax=300,
    ymin=0.5, ymax=4,
    legend pos=north west,
    ymajorgrids=true,
    grid style=dashed
]

\addplot[
    color=red,
    mark size=0.9pt
    ]
table[]
{img/txt/Zad1_Test/DMCu.txt};
 \addlegendentry{Sygnał sterujący $U(k)$};

\end{axis}
\end{tikzpicture}
%\fi

 % \iffalse
\begin{tikzpicture}
\begin{axis}[
     width=14cm,
    height=10cm,
    title={Rys 1.2},
    xlabel={Próbki},
    ylabel={Wartość},
    xmin=0, xmax=300,
    ymin=0.5, ymax=4,
    legend pos=north west,
    ymajorgrids=true,
    grid style=dashed
]

\addplot[
    color=blue,
    mark size=0.9pt
    ]
table[]
{img/txt/Zad1_Test/DMCy.txt};
 \addlegendentry{Wyjście procesu $Y(k)$};


\end{axis}
\end{tikzpicture}
%\fi

 \chapter{Odpowiedź skokowa procesu}
    \section{Implementacja}
        Implementacja zadania w środowisku Matlab jest w pliku \texttt{zad2\_3\_odpowiedzi\_skokowe}.
        
    \section{Przygotowanie odpowiedzi skokowej dla algorymu DMC}
        Zostało wyznaczonych symulacyjnie 5 odpowiedzi skokowych, o skokach z wartości $U=\num{1,5}$ na kolejno $U=\num{1,6}$, $U=\num{1,7}$, $U=\num{1,8}$, $U=\num{1,9}$, $U=\num{2,0}$.

        Uwzględniono ograniczenia sygnału w postaci $U_{\mathrm{min}}$ = 1, $U_{\mathrm{max}}$ = 2.
        

\begin{tikzpicture}
\begin{axis}[
     width=14cm,
    height=10cm,
    title={Odpowiedzi skokowe - Wyjście procesu Y(k)},
    xlabel={Próbki},
    ylabel={Wartość},
    xmin=0, xmax=300,
    ymin=2.1, ymax=2.6,
    legend pos=north east,
    ymajorgrids=true,
    grid style=dashed
]

\addplot[
    color=red,
    mark size=0.9pt
    ]
table[]
{img/txt/SKOKI_DMC/Y/skok i =6.txt};
 \addlegendentry{$\Delta$U = 0,5};
 \addplot[
    color=blue,
    mark size=0.9pt
    ]
table[]
{img/txt/SKOKI_DMC/Y/skok i =4.txt};
 \addlegendentry{$\Delta$U = 0,4};
 
 \addplot[
    color=orange,
    mark size=0.9pt
    ]
table[]
{img/txt/SKOKI_DMC/Y/skok i =3.txt};
 \addlegendentry{$\Delta$U = 0,3};
 
\addplot[
    color=purple,
    mark size=0.9pt
    ]
table[]
{img/txt/SKOKI_DMC/Y/skok i =2.txt};
 \addlegendentry{$\Delta$U = 0,2};
 
\addplot[
    color=brown,
    mark size=0.9pt
    ]
table[]
{img/txt/SKOKI_DMC/Y/skok i =1.txt};
 \addlegendentry{$\Delta$U = 0,1};
 
\end{axis}
\end{tikzpicture}

%%%%%%%%%%%%%%%%%%%%%%%%%%%%%%wejscie procesu%%%%%%%%%%%%%%%%%%%%%%%%%%%%%%%%%%

\begin{tikzpicture}
\begin{axis}[
     width=14cm,
    height=10cm,
    title={Odpowiedzi skokowe - Sygnał sterujący $U(k)$},
    xlabel={Próbki},
    ylabel={Wartość},
    xmin=0, xmax=300,
    ymin=1.4, ymax=2.1,
    legend pos=north east,
    ymajorgrids=true,
    grid style=dashed
]

\addplot[
    color=red,
    mark size=0.9pt
    ]
table[]
{img/txt/SKOKI_DMC/U/skok i =5.txt};
 \addlegendentry{$\Delta$U = 0,5};
 \addplot[
    color=blue,
    mark size=0.9pt
    ]
table[]
{img/txt/SKOKI_DMC/U/skok i =4.txt};
 \addlegendentry{$\Delta$U = 0,4};
 
 \addplot[
    color=orange,
    mark size=0.9pt
    ]
table[]
{img/txt/SKOKI_DMC/U/skok i =3.txt};
 \addlegendentry{$\Delta$U = 0,3};
 
\addplot[
    color=purple,
    mark size=0.9pt
    ]
table[]
{img/txt/SKOKI_DMC/U/skok i =2.txt};
 \addlegendentry{$\Delta$U = 0,2};
 
\addplot[
    color=brown,
    mark size=0.9pt
    ]
table[]
{img/txt/SKOKI_DMC/U/skok i =1.txt};
 \addlegendentry{$\Delta$U = 0,1};
 

\end{axis}
\end{tikzpicture}

    Z powyższych wykresów wynika, że odpowiedź skokowa obiektu zmienia się proporcjonalnie, wraz z wartościami skoku sygnału sterującego. Wskazuje to na liniowe w przybliżeniu właściwości dynamiczne procesu. 
    \newpage
    
    \section{Charakterystyka statyczna}
    Charakterystyka statyczna procesu widoczna na rysunku poniżej została wyznaczona poprzez symulowanie układu z różnymi wartościami sygnału wejściowego. Z wykresu widać że obiekt w przedziale $[1,2]$ ma w przybliżeniu właściwości statyczne. Wzmocnienie statyczne liczone jako współczynnik kierunkowy prostej zostało obliczone na $K_{\mathrm{stat}} = \num{0,77078}$.\paragraph{}
    

    \begin{tikzpicture}
            \begin{axis}[
             width=14cm,
            height=10cm,
            title={Charakterystyka statyczna},
            xlabel={U},
            ylabel={Y},
            xmin=-0, xmax=3,
            ymin=1.6, ymax=2.8,
            legend pos=north east,
            ymajorgrids=true,
            grid style=dashed
        ]
        
        \addplot[
            color=blue,
            mark size=0.9pt
            ]
        table[]
        {img/txt/Kstat/OUTy.txt};
     
        \end{axis}
\end{tikzpicture}
\chapter{Przekształcanie odpowiedzi skokowej i charakterystyka statyczna}
    \section{Implementacja}
    Implementacja zadania w środowisku Matlab jest w pliku \texttt{Zad2\_3}.
    \section{Przekształcanie odpowiedzi skokowej}
    Do wyznaczenia odpowiedzi skokowej użyto danych uzyskanych przy skoku wartości sterowania z wartości $\num{1,5}$ do $2$. Jako, że skok ten został wykonany w chwili dyskretnej $k=20$, seria danych wykorzystana do uzyskania odpowiedzi skokowej to wartości sygnału wyjściowego rejestrowane od chwili $k=21$. Przekształcanie odpowiedzi skokowej odbyło się z wykorzystaniem wzoru:
    
    \begin{equation}
    S_i=\frac{ {S_i-Y_{\mathrm{PP}}} }{\Delta{U}}\mathrm{,\;dla\;}{ i}\mathrm{\;= 1,...}
    \end{equation}
    
    , gdzie:\newline 
    $S_{i}$ - gotowa odpowiedź skokowa podawana jako model regulatora DMC, \newline
    $S_{i}^0$  - seria pomiarów pozyskanych w celu wyznaczenia odpowiedzi skokowej,czyli wartości sygnałów wyjściowych od chwili dyskretnej $k=21$ \newline
    $\Delta{U}$ - przyrost wartości sterowania, czyli $\num{0,5}$.\newline
    Odpowiedź skokowa jest widoczna na rysunku poniżej:



\begin{tikzpicture}
\begin{axis}[
     width=14cm,
    height=10cm,
    title={Odpowiedź skokowa},
    xlabel={Próbki},
    ylabel={Wartość},
    xmin=25, xmax=300,
    ymin=0, ymax=1,
    legend pos=north east,
    ymajorgrids=true,
    grid style=dashed
]

\addplot[
    color=red,
    mark size=0.9pt
    ]
table[]
{img/txt/SKOKI_DMC/Y/skok i =5.txt};
 \addlegendentry{$\Delta$U = 0,5};



\end{axis}
\end{tikzpicture}

\chapter{Implementacja regulatorów PID i DMC w Matlabie}
    \section{Implementacja}
    Implementacja zadania w środowsku Matlab jest w plikach \texttt{zad4\_PID} i \texttt{zad4\_DMC}.\paragraph{}
    
    Zgodnie z instrukcjami zawartymi w materiałach ćwiczeniowych w algorytmie jest zaimplementowane przesunięcie do punktu pracy $[0,0]$. Przed algorytmem regulacji dokonano podstawienia $Y = Y – Y_{\mathrm{PP}}$ oraz $Y_{\mathrm{zad}}=Y_{\mathrm{zad}}-Y_{\mathrm{PP}}$,
    natomiast po wyznaczeniu wartości sterowania dokonano podstawienia $U = U + U_{\mathrm{PP}}$, uprzednio uwzględniając ograniczenia według poniższego pseudokodu:
    \newline\newline
    jeżeli $\Delta{u}(k|k)<-\Delta{u^\mathrm{max}}${ to }
    $\Delta{u}(k|k)=-\Delta{u^\mathrm{max}}$
    \newline
    jeżeli $\Delta{u}(k|k)<\Delta{u^\mathrm{max}}${ to }
    $\Delta{u}(k|k)=\Delta{u^\mathrm{max}}$
    \newline
    $u(k|k)=\Delta{u}(k|k)+u(k-1)$
    \newline
    jeżeli $u(k|k)<u^\mathrm{min}${ to }
    $u(k|k)=u^\mathrm{min}$
    \newline
    jeżeli $u(k|k)>u^\mathrm{max}${ to }
    $u(k|k)=u^\mathrm{max}$   
    \newline
    $u(k)=u(k|k)$
    \newline
    
    Z racji tego, że ograniczenia uwzględniano przed dokonaniem podstawienia $U=U+U_{\mathrm{PP}}$, zostały one przesunięte przed algorytmem regulacji w następujący sposób:
    
    $U_{\mathrm{max}}=U_{\mathrm{max}}-U_{\mathrm{PP}}$, $U_{\mathrm{min}}=U_{\mathrm{min}}-U_{\mathrm{PP}}$.
        
    
    \section{Regulator PID}
    Regulator PID został zaimplementowany na podstawie wzorów:
    
    \begin{equation}
        u(k)=r_2e(k-2)+r_1e(k-1)+r_0e(k)+u(k-1)
    \end{equation}
    
    \begin{equation}
        r_2=\frac{KT_{\mathrm{d}}}{T},\; 
        r_1=K(\frac{T}{2T_{\mathrm{i}}}
            -2\frac{T_{\mathrm{d}}}{T}
            -1),\; 
        r_0=K(1+\frac{T}{2T_{\mathrm{i}}}
            +\frac{T_{\mathrm{d}}}{T})
    \end{equation}   
    
    Gdzie $e(k)$ to uchyb w chwiki $k$ wyrażony wzorem:
    
    \begin{equation}
       e(k)=y_{zad}(k)-y(k)
    \end{equation}
    
    \section{Regulator DMC}
    Regulator DMC został zaimplementowany na podstawie wzorów podanych na ćwiczeniach i udostępnionych na stronie przedmiotu.
    \newline
    Również tutaj zostało zaimplementowane przesunięcie do punktu pracy $[0,0]$ opisane w punkcie 4.2 .
    
    \section{Program do symulacji algorytmu DMC}
    Program działa w oparciu o następujący pseudokod:\newline\newline
    \texttt{INICJALIZACJA}\newline
    \texttt{Na podstawie odpowiedzi skokowej oblicz macierze: $M^P$, $M$, $K$}\newline\newline
    \texttt{DLA KAŻDEJ CHWILI k:}\newline
    \texttt{1. Pomiar $y(k)$}\newline
    \texttt{2. Oblicz macierze: $\Delta{U^P(k)}$,$Y(k)$,$Y^0(k)$}\newline
    \texttt{3. Oblicz: $\Delta{U(k)}=K(Y^{\mathrm{zad}}(k)-Y^0(k))$}\newline
    \texttt{4. Zastosuj do sterowania pierwszy element $\Delta{U(k)}$: $u(k)=\Delta{u}(k|k)+u(k-1)$}\newline
    \paragraph{}
    Wzory na powyższe macierze zostały zaczerpnięte z materiałów do ćwiczeń.
    
    
   
    
\chapter{Dobór nastaw i parametrów regulatorów PID i DMC metodą eksperymentalną}
    \section{Implementacja}
    Implementacja zadania w środowsku Matlab jest w plikach \texttt{Zad4\_PID} i \texttt{Zad4\_DMC}.
    \section{Dobór nastaw regulatora PID metodą eksperymentalną.}
    Dobór nastaw odbył się metodą „inżynierską” tzn. obiekt regulacji został wprowadzony w oscylacje nierosnące i niegasnące za pomocą odpowiedniej wartości wzmocnienia. Takie wzmocnie nazywamy krytycznym i w tym przypadku było równe $K_{\mathrm{kryt}}=\num{5,03}$. 
    
          %%%%%%%%%%%%%%%%%%%%%%%%%%%%%%%%%%%%%%%%%    %K=5,03 -Kkryt
\begin{tikzpicture}
        \begin{axis}[
             width=14cm,
            height=10cm,
            title={PID $K_{\mathrm{kryt}}=\num{5,03}$},
            xlabel={Próbki},
            ylabel={Wartość},
            xmin=0, xmax=1400,
            ymin=2, ymax=2.6,
            legend pos=north west,
            ymajorgrids=true,
            grid style=dashed
        ]
        
        \addplot[
            color=blue,
            mark size=0.9pt
            ]
        table[]
        {img/txt/PID/PIDKkryt=5,3/OUTy.txt};
         \addlegendentry{Wyjście procesu $Y(K)$};
        
        \addplot[
                color=purple,
                dotted,
                mark=*,
                mark options={solid},
                smooth,
                mark size=0.05pt
            ]
        table[]
        {img/txt/PID/PIDKkryt=5,3/OUTyzad.txt};
        
        \addlegendentry{Wartość zadana $Y_{\mathrm{zad}}(k)$};
        
        \end{axis}
\end{tikzpicture}


    \paragraph{}
    Mnożąc go przez $\num{0,5}$ uzyskaliśmy człon proporcjonalny. Potem dobraliśmy taki czas zdwojenia $T_{\mathrm{i}}$, by uzyskać zadowalającą wielkość wskaźnika jakości regulacji. Na koniec dobraliśmy taki czas wyprzedzenia $T_{\mathrm{d}}$, żeby wskaźnik jakości regulacji był jak najmniejszy.
    \paragraph{}
    {  }W obu algorytmach liczony był wskaźnik jakości regulacji ze wzoru:
   
    \begin{equation}
        E=\sum_{k=1}^{k_{\mathrm{konc}}}(y^{\mathrm{zad}}(k)-y(k))^2
    \end{equation}
    
    Ostateczne wartości parametrów dobranych metodą inżynierską:
    $K=\num{2,515}$, $T_{\mathrm{i}}=22$, $T_{\mathrm{d}}=7$
    Wartość wskaźnika jakości regulacji wyniosła $E= \num{17,1744}$
   
   %strojenie inzynierksie 
    
\begin{tikzpicture}
        \begin{axis}[
             width=14cm,
            height=10cm,
            title={PID Strojenie metodą eksperymentalną},
            xlabel={Próbki},
            ylabel={Wartość},
            xmin=0, xmax=1400,
            ymin=1.6, ymax=2.8,
            legend pos=north west,
            ymajorgrids=true,
            grid style=dashed
        ]
        
        \addplot[
            color=blue,
            mark size=0.9pt
            ]
        table[]
        {img/txt/PID/PID_Strojenie_INZ/Skoki/OUTy.txt};
         \addlegendentry{Wyjście procesu $Y(k)$};
        \addplot[
                color=purple,
                dotted,
                mark=*,
                mark options={solid},
                smooth,
                mark size=0.05pt
            ]
        table[]
        {img/txt/PID/PID_Strojenie_INZ/Skoki/OUTyzad.txt};
        
        \addlegendentry{Wartość zadana $Y_{\mathrm{zad}}(k)$};
        
        \end{axis}
\end{tikzpicture}
    
      
\begin{tikzpicture}
        \begin{axis}[
             width=14cm,
            height=10cm,
            title={PID Strojenie metodą eksperymentalną},
            xlabel={Próbki},
            ylabel={Wartość},
            xmin=0, xmax=1400,
           ymin=1, ymax=2.5,
            legend pos=north west,
            ymajorgrids=true,
            grid style=dashed
        ]
        \addplot[
                color=red,
                mark size=0.9pt
                ]
        table[]
        {img/txt/PID/PID_Strojenie_INZ/Skoki/OUTu.txt};
         \addlegendentry{Sygnał sterujący $U(k)$};
        \end{axis}
\end{tikzpicture}
    \newpage

    \section{Dobór parametrów algorytmu DMC metodą eksperymentalną}
    Na początku przy dużych wartościach parametrów $N=50$ oraz $N_{\mathrm{u}}=15$ i parametrze lambda=1 zmniejszano wartość parametru $D$ od wartości $D=200$ do $D=50$. 
    \newline\newline
    %%%%%%%%%%%%%%%%start
\begin{tikzpicture}
            \begin{axis}[
             width=14cm,
            height=10cm,
            title={DMC  Początkowe Wartości},
            xlabel={Próbki},
            ylabel={Wartość},
            xmin=0, xmax=1400,
            ymin=1.8, ymax=2.6,
            legend pos=north east,
            ymajorgrids=true,
            grid style=dashed
        ]
        
        \addplot[
            color=blue,
            mark size=0.9pt
            ]
        table[]
        {img/txt/DMC/DMC_Start/OUTy.txt};
         \addlegendentry{Wyjście procesu $Y(k)$};

        \addplot[
                color=purple,
                dotted,
                mark=*,
                mark options={solid},
                smooth,
                mark size=0.05pt
            ]
        table[]
        {img/txt/DMC/DMC_Start/OUTyzad.txt};
        
        \addlegendentry{Wartość zadana $Y_{\mathrm{zad}}(k)$};
        
        \end{axis}
\end{tikzpicture}


\begin{tikzpicture}
            \begin{axis}[
             width=14cm,
            height=10cm,
            title={DMC początkowe wartości parametrów},
            xlabel={Próbki},
            ylabel={Wartość},
            xmin=0, xmax=1400,
            ymin=0.8, ymax=2.2,
            legend pos=north east,
            ymajorgrids=true,
            grid style=dashed
        ]
        \addplot[
                color=red,
                mark size=0.9pt
                ]
        table[]
        {img/txt/DMC/DMC_Start/OUTu.txt};
         \addlegendentry{Sygnał sterujący U(k)};

        
        \end{axis}
\end{tikzpicture}

    %%%%%%%%%%   D
\begin{tikzpicture}
        \begin{axis}[
             width=14cm,
            height=10cm,
            title={DMC optymalna wartość D},
            xlabel={Próbki},
            ylabel={Wartość},
            xmin=0, xmax=1400,
            ymin=1.8, ymax=2.6,
            legend pos=north east,
            ymajorgrids=true,
            grid style=dashed
        ]
        
        \addplot[
            color=blue,
            mark size=0.9pt
            ]
        table[]
        {img/txt/DMC/DMC_D_op/OUTy.txt};
         \addlegendentry{Wyjście procesu $Y(k)$};
        \addplot[
                color=purple,
                dotted,
                mark=*,
                mark options={solid},
                smooth,
                mark size=0.05pt
            ]
        table[]
        {img/txt/DMC/DMC_D_op/OUTyzad.txt};
        
        \addlegendentry{Wartość zadana $Y_{\mathrm{zad}}(k)$};
        
        \end{axis}
\end{tikzpicture}


\begin{tikzpicture}
        \begin{axis}[
             width=14cm,
            height=10cm,
            title={DMC optymalne D},
            xlabel={Próbki},
            ylabel={Wartość},
            xmin=0, xmax=1400,
            ymin=0.8, ymax=2.2,
            legend pos=north east,
            ymajorgrids=true,
            grid style=dashed
        ]

        \addplot[
                color=red,
                mark size=0.9pt
                ]
        table[]
        {img/txt/DMC/DMC_D_op/OUTu.txt};
         \addlegendentry{Sygnał sterujący $U(k)$};

        
        \end{axis}
\end{tikzpicture}

%%%%%%%%%%%% N


\begin{tikzpicture}
        \begin{axis}[
             width=14cm,
            height=10cm,
            title={DMC optymalne $N$},
            xlabel={Próbki},
            ylabel={Wartość},
            xmin=0, xmax=1400,
            ymin=1.8, ymax=2.6,
            legend pos=north east,
            ymajorgrids=true,
            grid style=dashed
        ]
        
        \addplot[
            color=blue,
            mark size=0.9pt
            ]
        table[]
        {img/txt/DMC/DMC_N_op/OUTy.txt};
         \addlegendentry{Wyjście procesu $Y(k)$};
        \addplot[
                color=purple,
                dotted,
                mark=*,
                mark options={solid},
                smooth,
                mark size=0.05pt
            ]
        table[]
        {img/txt/DMC/DMC_N_op/OUTyzad.txt};
                 \addlegendentry{Wartość zadana $Y_{\mathrm{zad}}(k)$};
        
        \end{axis}
\end{tikzpicture}



\begin{tikzpicture}
        \begin{axis}[
             width=14cm,
            height=10cm,
            title={DMC optymalne N},
            xlabel={Próbki},
            ylabel={Wartość},
            xmin=0, xmax=1400,
            ymin=0.8, ymax=2.2,
            legend pos=north east,
            ymajorgrids=true,
            grid style=dashed
        ]
        

         
        \addplot[
                color=red,
                mark size=0.9pt
                ]
        table[]
        {img/txt/DMC/DMC_N_op/OUTu.txt};
         \addlegendentry{Sygnał sterujący $U(k)$};
    
        \end{axis}
\end{tikzpicture}



%%%%%%%%%%%% NU

\begin{tikzpicture}
        \begin{axis}[
             width=14cm,
            height=10cm,
            title={DMC optymalne Nu},
            xlabel={Próbki},
            ylabel={Wartość},
            xmin=0, xmax=1400,
            ymin=1.8, ymax=2.6,
            legend pos=north east,
            ymajorgrids=true,
            grid style=dashed
        ]
        
        \addplot[
            color=blue,
            mark size=0.9pt
            ]
        table[]
        {img/txt/DMC/DMC_Nu_op/OUTy.txt};
         \addlegendentry{Wyjście procesu $Y(k)$};
         

        \addplot[
                color=purple,
                dotted,
                mark=*,
                mark options={solid},
                smooth,
                mark size=0.05pt
            ]
        table[]
        {img/txt/DMC/DMC_Nu_op/OUTyzad.txt};
         \addlegendentry{Wartość zadana $Y_{\mathrm{zad}}(k)$};
        
        \end{axis}
\end{tikzpicture}


\begin{tikzpicture}
        \begin{axis}[
             width=14cm,
            height=10cm,
            title={DMC optymalne $N_{\mathrm{u}}$},
            xlabel={Próbki},
            ylabel={Wartość},
            xmin=0, xmax=1400,
            ymin=0.8, ymax=2.2,
            legend pos=north east,
            ymajorgrids=true,
            grid style=dashed
        ]
        
        \addplot[
                color=red,
                mark size=0.9pt
                ]
        table[]
        {img/txt/DMC/DMC_Nu_op/OUTu.txt};
         \addlegendentry{Sygnał sterujący $U(k)$};

        
        \end{axis}
\end{tikzpicture}


   
    Starano się uzyskać możliwie małą wartość tego parametru, przy której regulacja przebiegała dobrze (tzn. po ustabilizowaniu się na odpowiedniej wartości zadanej po pewnym czasie nie było widocznego skoku wartości sygnału wyjściowego), a przy tym wskaźnik jakości był wystarczająco mały. Kolejnym strojonym parametrem był parametr $N$, który również był stopniowo zmniejszany i oceniany pod względem jakości regulacji. Analogicznie postąpiono z parametrem $N_{\mathrm{u}}$. Parametr lambda natomiast był stopniowo zwiększany, lecz jego wzrost nie implikował wzrostu jakości regulacji.
    
    
    Ostatecznie optymalne parametry uzyskane eksperymentalnie wyniosły:
    $D=150$; $N=35$, $N_{\mathrm{u}}=5$, $\lambda=1$.
    Wartość wskaźnika jakości regulacji wyniosła: $E= \num{8,5954}$.
 
   %punkt startowy strojenia inz
    
  %%%%%%%%%%%% lamb

\begin{tikzpicture}
        \begin{axis}[
             width=14cm,
            height=10cm,
            title={DMC optymalne wartości wszystkich parametrów},
            xlabel={Próbki},
            ylabel={Wartość},
            xmin=0, xmax=1400,
            ymin=1.8, ymax=2.6,
            legend pos=north east,
            ymajorgrids=true,
            grid style=dashed
        ]
        
        \addplot[
            color=blue,
            mark size=0.9pt
            ]
        table[]
        {img/txt/DMC/DMC_lamb_op/OUTy.txt};
         \addlegendentry{Wyjście procesu Y(k)};
        \addplot[
                color=purple,
                dotted,
                mark=*,
                mark options={solid},
                smooth,
                mark size=0.05pt
            ]
        table[]
        {img/txt/DMC/DMC_lamb_op/OUTyzad.txt};
                 \addlegendentry{Wartość zadana $Y_{\mathrm{zad}}(k)$};
        
        \end{axis}
\end{tikzpicture}



\begin{tikzpicture}
        \begin{axis}[
             width=14cm,
            height=10cm,
            title={DMC optymalne wartości wszystkich parametrów},
            xlabel={Próbki},
            ylabel={Wartość},
            xmin=0, xmax=1400,
            ymin=0.8, ymax=2.2,
            legend pos=north east,
            ymajorgrids=true,
            grid style=dashed
        ]
        \addplot[
                color=red,
                mark size=0.9pt
                ]
        table[]
        {img/txt/DMC/DMC_lamb_op/OUTu.txt};
         \addlegendentry{Sygnał sterujący $U(k)$};

        \end{axis}
\end{tikzpicture}
\chapter{Dobór nastaw i parametrów regulatorów PID i DMC metodą analityczną}
    \section{Implementacja}
    Implementacja w środowisku Matlab znajduje się w \texttt{plikach PID\_optymalizacja}, \texttt{DMC\_optymalizacja}, \texttt{PID\_fun.m}, \texttt{DMC\_fun}.
    
    Na początku wcześniej napisane algorytmy PID oraz DMC zostały przeniesione do następujących funkcji:
    \newline
    function wskaznik\_jakosci=PID\_fun(parameters)
    \newline
    function wskaznik\_jakosci=DMC\_fun(parameters)
    \newline
    , które jak widać jako argumenty przyjmują parametry regulacji, a zwracają wartość wskaźnika jakości regulacji. Następnie w plikach PID\_optymalizacja.m, DMC\_optymalizacja.m przekazano powyższe funkcje do funkcji optymalizującej programu MATLAB o nazwie fmincon. W przypadku regulatora PID ograniczenia dolne wyglądały następująco:
    
    \begin{equation}
        K\geqslant{\num{0,1}},\;  T_{\mathrm{i}}\geqslant{\num{0,1}},\; T_{\mathrm{d}}\geqslant{\num{0,1}} 
    \end{equation}
    
    natomiast w przypadku regulatora DMC:
    \begin{equation}
 	    N\geqslant{1},\; N_{\mathrm{u}}\geqslant{1},\; \lambda\geqslant{1}
    \end{equation}
    
    Ograniczeń górnych nie uwzględniono. Jako wartości początkowe przyjęto wartości wyznaczone eksperymentalnie. W przypadku regulatora DMC wprowadzono restrykcję sprawiającą, że obliczone parametry muszą być całkowitoliczbowe ze względu na to, że rozmiar macierzy tworzonych w algorytmie regulacji DMC od nich zależy, a z oczywistych względów rozmiar macierzy nie może być liczbą niecałkowitą.
    \paragraph{}
    Po wykonaniu optymalizacji funkcją fmincon() nastawy obu algorytmów są zapisywane jako przestrzeń robocza programu MATLAB w plikach o nazwach:
    \newline\newline
    \texttt{optymalne\_parametry\_DMC.mat}
    \newline
    \texttt{optymalne\_parametry\_PID.mat}
    \paragraph{} 
    W zmiennych nazywających się odpowiednio:
    \newline\newline
    \texttt{nastawy\_DMC\_fmincon}
    \newline
    \texttt{nastawy\_PID\_fmincon}
    \newline\newline

    \newpage
    \section{Wyniki optymalizacji}
    Parametry oraz wskaźniki uzyskane w wyniku optymalizacji wskaźnika jakości regulacji E przedstawiają się następująco:
    \newline\newline
    Regulator PID:
       %  \iffalse %komentarz
    \begin{center}
        \textsc{$K=\num{6,0667}$, $T_{\mathrm{i}}=\num{7,1952}$, $T_{\mathrm{d}}=\num{3,8373}$, $E=$\num{9,2884}}  
    \end{center}
    
\begin{tikzpicture}
        \begin{axis}[
             width=14cm,
            height=10cm,
            title={PID Strojenie przy użyciu funkcji fmincon},
            xlabel={Próbki},
            ylabel={Wartość},
            xmin=0, xmax=1400,
            ymin=1.8, ymax=2.8,
            legend pos=north east,
            ymajorgrids=true,
            grid style=dashed
        ]
        
        \addplot[
            color=blue,
            mark size=0.9pt
            ]
        table[]
        {img/txt/PID/PIDStrojeniefmincon/Skoki/OUTy.txt};
         \addlegendentry{Wyjście procesu $Y(k)$};
 
        
        \addplot[
                color=purple,
                dotted,
                mark=*,
                mark options={solid},
                smooth,
                mark size=0.05pt
            ]
        table[]
        {img/txt/PID/PIDStrojeniefmincon/Skoki/OUTyzad.txt};
        
        \addlegendentry{Wartość zadana $Y_{\mathrm{zad}}(k)$};
        
        \end{axis}
\end{tikzpicture}
      
      
\begin{tikzpicture}
        \begin{axis}[
             width=14cm,
            height=10cm,
            title={PID Strojenie przy użyciu funkcji fmincon},
            xlabel={Próbki},
            ylabel={Wartość},
            xmin=0, xmax=1400,
            ymin=0.8, ymax=2.5,
            legend pos=north east,
            ymajorgrids=true,
            grid style=dashed
        ]

        \addplot[
                color=red,
                mark size=0.9pt
                ]
        table[]
        {img/txt/PID/PIDStrojeniefmincon/Skoki/OUTu.txt};
         \addlegendentry{Sygnał sterujący $U(k)$};
    \end{axis}
\end{tikzpicture}
    
    \newpage
    Regulator DMC:
    \begin{center}
        \textsc{$N=51$, $N_{\mathrm{u}}=144$, $\lambda=1$, $E=8,5711$}  
    \end{center}
    
\begin{tikzpicture}
        \begin{axis}[
             width=14cm,
            height=10cm,
            title={DMC Strojenie przy użyciu funkcji fmincon},
            xlabel={Próbki},
            ylabel={Wartość},
            xmin=0, xmax=1400,
            ymin=1.8, ymax=2.8,
            legend pos=north east,
            ymajorgrids=true,
            grid style=dashed
        ]
        
        \addplot[
            color=blue,
            mark size=0.9pt
            ]
        table[]
        {img/txt/DMC/DMC_Fmin/OUTy.txt};
         \addlegendentry{Wyjście procesu $Y(k)$};

        \addplot[
                color=purple,
                dotted,
                mark=*,
                mark options={solid},
                smooth,
                mark size=0.05pt
            ]
        table[]
        {img/txt/DMC/DMC_Fmin/OUTyzad.txt};
        \addlegendentry{Wartość zadana $Y_{\mathrm{zad}}(k)$};
        
        \end{axis}
\end{tikzpicture}

    
\begin{tikzpicture}
        \begin{axis}[
             width=14cm,
            height=10cm,
            title={DMC Strojenie przy użyciu funkcji fmincon},
            xlabel={Próbki},
            ylabel={Wartość},
            xmin=0, xmax=1400,
            ymin=0.8, ymax=2.4,
            legend pos=north east,
            ymajorgrids=true,
            grid style=dashed
        ]
        \addplot[
                color=red,
                mark size=0.9pt
                ]
        table[]
        {img/txt/DMC/DMC_Fmin/OUTu.txt};
         \addlegendentry{Sygnał sterujący $U(k)$};

        \end{axis}
\end{tikzpicture}

    \newpage
    \section{Wnioski}
    Jak widać, w przypadku regulatora PID mimo że wskaźnik jakości regulacji jest znacznie niższy niż w przypadku eksperymentalnie dobranych parametrów, to występują duże oscylacje - obiekt nie stabilizuje się po osiągnięciu wartości zadanej. Oznacza to, że te parametry są zdecydowanie gorsze od parametrów eksperymentlanych i należałoby z tego wyciągnąć wniosek, że przy strojeniu regulatora PID nie należy kierować się tylko i wyłącznie sumą kwadratów uzyskanych uchybów.
    \paragraph{}
    W przypadku regulatora DMC dobrane parametry zapewniają mniejszą wartość wskaźnika E oraz dobrą jakość regulacji. Jest ona porównywalna do jakości regulacji otrzymanej przy parametrach dobranych eksperymentalnie, choć należy wspomnieć, że w wersji zoptymalizowanej występują nieznacznie większe przeregulowania. 

  \iffalse
\begin{tikzpicture}
\begin{axis}[
     width=14cm,
    height=10cm,
    title={DMC D=15, lambda=N=50, Nu=20},
    xlabel={Czas},
    ylabel={Wartość},
    xmin=0, xmax=400,
    ymin=0.5, ymax=4,
    legend pos=north west,
    ymajorgrids=true,
    grid style=dashed
]

\addplot[
    color=blue,
    mark size=0.9pt
    ]
table[]
{img/txt/BadDMC/DMCy.txt};
 \addlegendentry{Wyjście procesu Y};
 
\addplot[
        color=red,
        mark size=0.9pt
        ]
table[]
{img/txt/BadDMC/DMCu.txt};
 \addlegendentry{Wyjście regulatora U};

\addplot[
        color=purple,
        dotted,
        mark=*,
        mark options={solid},
        smooth,
        mark size=0.05pt
    ]
table[]
{img/txt/BadDMC/DMCyzad.txt};

\addlegendentry{Wartość zadana Yzad};

\end{axis}
\end{tikzpicture}
\fi

    

    
    

\end{document}
