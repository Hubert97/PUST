\chapter{Dobór nastaw i parametrów regulatorów PID i DMC metodą analityczną}
    \section{Implementacja}
    Implementacja w środowisku Matlab znajduje się w \texttt{plikach PID\_optymalizacja}, \texttt{DMC\_optymalizacja}, \texttt{PID\_fun.m}, \texttt{DMC\_fun}.
    
    Na początku wcześniej napisane algorytmy PID oraz DMC zostały przeniesione do następujących funkcji:
    \newline
    function wskaznik\_jakosci=PID\_fun(parameters)
    \newline
    function wskaznik\_jakosci=DMC\_fun(parameters)
    \newline
    , które jak widać jako argumenty przyjmują parametry regulacji, a zwracają wartość wskaźnika jakości regulacji. Następnie w plikach PID\_optymalizacja.m, DMC\_optymalizacja.m przekazano powyższe funkcje do funkcji optymalizującej programu MATLAB o nazwie fmincon. W przypadku regulatora PID ograniczenia dolne wyglądały następująco:
    
    \begin{equation}
        K\geqslant{\num{0,1}},\;  T_{\mathrm{i}}\geqslant{\num{0,1}},\; T_{\mathrm{d}}\geqslant{\num{0,1}} 
    \end{equation}
    
    natomiast w przypadku regulatora DMC:
    \begin{equation}
 	    N\geqslant{1},\; N_{\mathrm{u}}\geqslant{1},\; \lambda\geqslant{1}
    \end{equation}
    
    Ograniczeń górnych nie uwzględniono. Jako wartości początkowe przyjęto wartości wyznaczone eksperymentalnie. W przypadku regulatora DMC wprowadzono restrykcję sprawiającą, że obliczone parametry muszą być całkowitoliczbowe ze względu na to, że rozmiar macierzy tworzonych w algorytmie regulacji DMC od nich zależy, a z oczywistych względów rozmiar macierzy nie może być liczbą niecałkowitą.
    \paragraph{}
    Po wykonaniu optymalizacji funkcją fmincon() nastawy obu algorytmów są zapisywane jako przestrzeń robocza programu MATLAB w plikach o nazwach:
    \newline\newline
    \texttt{optymalne\_parametry\_DMC.mat}
    \newline
    \texttt{optymalne\_parametry\_PID.mat}
    \paragraph{} 
    W zmiennych nazywających się odpowiednio:
    \newline\newline
    \texttt{nastawy\_DMC\_fmincon}
    \newline
    \texttt{nastawy\_PID\_fmincon}
    \newline\newline

    \newpage
    \section{Wyniki optymalizacji}
    Parametry oraz wskaźniki uzyskane w wyniku optymalizacji wskaźnika jakości regulacji E przedstawiają się następująco:
    \newline\newline
    Regulator PID:
       %  \iffalse %komentarz
    \begin{center}
        \textsc{$K=\num{6,0667}$, $T_{\mathrm{i}}=\num{7,1952}$, $T_{\mathrm{d}}=\num{3,8373}$, $E=$\num{9,2884}}  
    \end{center}
    
\begin{tikzpicture}
        \begin{axis}[
             width=14cm,
            height=10cm,
            title={PID Strojenie przy użyciu funkcji fmincon},
            xlabel={Próbki},
            ylabel={Wartość},
            xmin=0, xmax=1400,
            ymin=1.8, ymax=2.8,
            legend pos=north east,
            ymajorgrids=true,
            grid style=dashed
        ]
        
        \addplot[
            color=blue,
            mark size=0.9pt
            ]
        table[]
        {img/txt/PID/PIDStrojeniefmincon/Skoki/OUTy.txt};
         \addlegendentry{Wyjście procesu $Y(k)$};
 
        
        \addplot[
                color=purple,
                dotted,
                mark=*,
                mark options={solid},
                smooth,
                mark size=0.05pt
            ]
        table[]
        {img/txt/PID/PIDStrojeniefmincon/Skoki/OUTyzad.txt};
        
        \addlegendentry{Wartość zadana $Y_{\mathrm{zad}}(k)$};
        
        \end{axis}
\end{tikzpicture}
      
      
\begin{tikzpicture}
        \begin{axis}[
             width=14cm,
            height=10cm,
            title={PID Strojenie przy użyciu funkcji fmincon},
            xlabel={Próbki},
            ylabel={Wartość},
            xmin=0, xmax=1400,
            ymin=0.8, ymax=2.5,
            legend pos=north east,
            ymajorgrids=true,
            grid style=dashed
        ]

        \addplot[
                color=red,
                mark size=0.9pt
                ]
        table[]
        {img/txt/PID/PIDStrojeniefmincon/Skoki/OUTu.txt};
         \addlegendentry{Sygnał sterujący $U(k)$};
    \end{axis}
\end{tikzpicture}
    
    \newpage
    Regulator DMC:
    \begin{center}
        \textsc{$N=51$, $N_{\mathrm{u}}=144$, $\lambda=1$, $E=8,5711$}  
    \end{center}
    
\begin{tikzpicture}
        \begin{axis}[
             width=14cm,
            height=10cm,
            title={DMC Strojenie przy użyciu funkcji fmincon},
            xlabel={Próbki},
            ylabel={Wartość},
            xmin=0, xmax=1400,
            ymin=1.8, ymax=2.8,
            legend pos=north east,
            ymajorgrids=true,
            grid style=dashed
        ]
        
        \addplot[
            color=blue,
            mark size=0.9pt
            ]
        table[]
        {img/txt/DMC/DMC_Fmin/OUTy.txt};
         \addlegendentry{Wyjście procesu $Y(k)$};

        \addplot[
                color=purple,
                dotted,
                mark=*,
                mark options={solid},
                smooth,
                mark size=0.05pt
            ]
        table[]
        {img/txt/DMC/DMC_Fmin/OUTyzad.txt};
        \addlegendentry{Wartość zadana $Y_{\mathrm{zad}}(k)$};
        
        \end{axis}
\end{tikzpicture}

    
\begin{tikzpicture}
        \begin{axis}[
             width=14cm,
            height=10cm,
            title={DMC Strojenie przy użyciu funkcji fmincon},
            xlabel={Próbki},
            ylabel={Wartość},
            xmin=0, xmax=1400,
            ymin=0.8, ymax=2.4,
            legend pos=north east,
            ymajorgrids=true,
            grid style=dashed
        ]
        \addplot[
                color=red,
                mark size=0.9pt
                ]
        table[]
        {img/txt/DMC/DMC_Fmin/OUTu.txt};
         \addlegendentry{Sygnał sterujący $U(k)$};

        \end{axis}
\end{tikzpicture}

    \newpage
    \section{Wnioski}
    Jak widać, w przypadku regulatora PID mimo że wskaźnik jakości regulacji jest znacznie niższy niż w przypadku eksperymentalnie dobranych parametrów, to występują duże oscylacje - obiekt nie stabilizuje się po osiągnięciu wartości zadanej. Oznacza to, że te parametry są zdecydowanie gorsze od parametrów eksperymentlanych i należałoby z tego wyciągnąć wniosek, że przy strojeniu regulatora PID nie należy kierować się tylko i wyłącznie sumą kwadratów uzyskanych uchybów.
    \paragraph{}
    W przypadku regulatora DMC dobrane parametry zapewniają mniejszą wartość wskaźnika E oraz dobrą jakość regulacji. Jest ona porównywalna do jakości regulacji otrzymanej przy parametrach dobranych eksperymentalnie, choć należy wspomnieć, że w wersji zoptymalizowanej występują nieznacznie większe przeregulowania. 